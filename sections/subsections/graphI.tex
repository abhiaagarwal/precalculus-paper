\subsection{Graph I}

Graph I is ${y=\arccsc(x)}$.
Arccosecant or Inverse Cosecant gives an output whose angle resides in Quadrants I and IV, and is the inverse function of ${y=\csc(x)}$.
The domain must be restricted as the original function ${y=\csc(x)}$ fails the horizontal line test and is therefore not one-to-one.
The function can be restricted to the period of ${\pi}$.
${y=\arccsc(x)}$ has “branches”, essentially separate parts of the function which are each period.
The best interval is the domain ${x \in {[\frac{{-pi}}{2},\frac{pi}{2}]}}$.
This is because the function is asymptotic x=0 and all the integer multiples of pi, which is evident in this graph.
In addition, the range of ${y=\csc(x)}$ is ${y \in {{({-\infty}, {-1})} \cup {(1, {\infty})}}}$, and the domain of this function appears to be ${x \in {{({-\infty}, {-1})} \cup {(1, {\infty})}}}$.
The range of this function reflects the domain of its restricted counterpart ${y=\csc(x)}$; it is ${x \in {[\frac{{-pi}}{2},\frac{pi}{2}]}, {x \neq 0}}$, while this graph has a range of ${y \in {[\frac{{-pi}}{2},\frac{pi}{2}]}, {y \neq 0}}$.
