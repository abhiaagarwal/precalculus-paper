\subsection{Graph F}

Graph F is ${y=\arctan(x)}$.
Arctangent or Inverse Tangent takes a ratio and returns an angle in Quadrant I and IV and is the inverse function of ${y=\tan(x)}$.
The original function must be restricted as ${y=\tan(x)}$ is not an one-to-one function and therefore fails the horizontal line test.
The domain must be restricted to the length of one period, $\pi$, as that retains all the outputs of the graph.
There are an infinite amount of ways that this can be done; however, the best choice is ${x \in {(\frac{-\pi}{2}, \frac{\pi}{2})}}$.
This is because the function is continuous on this domain, avoiding the asymptotes that happen on every odd integer multiple of ${\frac{\pi}{2}}$, and is the closest domain on the y-axis.
Inverse functions swap the domain and range with the original function.
The domain and range of the restricted ${y=\tan(x)}$ are ${x \in {(\frac{-\pi}{2}, \frac{\pi}{2})}}$ and ${y \in \R}$, while those are the range and domain of the inverse function respectively.
This is shown on the Graph.
Since arctangent takes a ratio, it is reasonable that the function outputs an angle between ${[\frac{-\pi}{2}, \frac{\pi}{2}]}$, as adding a period will give an equivalent ratio along the circle.
