\subsection{Graph J}

Graph J is ${y=\csc(x)}$.
Cosecant is defined as the reciprocal of ${y=\sin(x)}$ (${\frac{1}{\sin(x)}}$) and takes an angle and outputs a ratio.
Since ${\sin(n\pi)}$ for ${n \in \Z}$, at those points, ${{\csc(n\pi)} \in \emptyset}$.
Therefore, the function is asymptotic at those points, which is evident on this graph.
The period of this graph appears to be ${6.28}$, which approximates ${2\pi}$.
This is the period of sine and therefore cosecant.
Since ${y=\sin(x)}$ starts at zero and ends at zero, the asymptotes of ${y = \csc(x)}$ happen at the start, middle, and the end of the period, while the relative maximum and minimum for each branch are ${\frac{1}{4}}$ths in and ${\frac{3}{4}}$ths in, which is shown on the given graph.
This function has two distinct branches, one which approaches positive infinity while the other approaches negative infinity, which is relevant on the graph.
