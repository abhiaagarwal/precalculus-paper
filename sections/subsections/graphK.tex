\subsection{Graph K}

Graph K is ${y=\cot(x)}$.
Tangent is defined as the ratio of Cosine and Sine, and takes an angle and outputs a ratio.
It is also can be defined as the reciprocal of tangent.
The function appears to have a period of $3.14$, which is approximately $\pi$, a property of the function.
Since the function is defined as ${\frac{\cos(x)}{\sin(x)}}$, and ${\sin(n\pi)=0}$ for ${n \in \Z}$, the function is asymptotic at ${x=n\pi}$.
${\cos(n\frac{\pi}{2})=0}$ for ${\text{n is odd}, }$, so ${\cot(n\frac{\pi}{2})=0}$, so the x-intercepts of cotangent are all odd integer multiples of ${\frac{\pi}{2}}$.
This is shown on the graph.
During the first half of its period, ${\sin(x)}$ is positive and ${\cos(x)}$ is positive, and ${\sin(x)}$ grows to $1$ from $0$ while ${\cos(x)}$ shrinks to $0$ from $1$, so the function approaches $0$.
During the second half of its period, ${\sin(x)}$ is positive and ${\cos(x)}$ is negative, and ${\sin(x)}$ shrinks to $0$ while ${\cos(x)}$ grows to $-1$ from $0$, meaning the function approaches infinity.
This process repeats every $\pi$ radians.
