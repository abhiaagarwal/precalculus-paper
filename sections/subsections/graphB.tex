\subsection{Graph B}

Graph B is $y=\arcsec(x)$.
Arcsecant or Inverse Secant gives a positive output whose angle resides in Quadrants I and II, and is the inverse function of $y=\sec(x)$.
The domain must be restricted as the original function $y=\sec(x)$ fails the horizontal line test.
The function can be restricted to the length of half a period of $\pi$.
This is possible as $y=\arcsec(x)$ has “branches”, essentially separate parts of the function which are each period.
There is only one way this can be done without failing the horizontal line test, and this is the domain $x \in{[0,\pi]}$.
This is because the function is asymptotic with $x=\frac{\pi}{2}$, which is evident in Graph B.
In addition, the range of $y=\sec(x)$ is $y \in {{({-\infty}, {-1})} \cup {(1, {\infty})}}$, and the domain of this function appears to be $x \in {{({-\infty}, {-1})} \cup {(1, {\infty})}}$.
In addition, the range of this function reflects the domain of its restricted counterpart $y=\sec(x)$; it is ${x \in {[0, \pi]}}, {x \neq \frac{\pi}{2}}$, while this graph has a range of ${y \in {[0, \pi]}}, {y \neq \frac{\pi}{2}}$.
