\subsection{Graph D}

Graph D is ${y=\arcsin(x)}$.
Arcsine or Inverse Sine is a function which takes a ratio as an input and outputs an angle.
Since ${y=\sin(x)}$ fails the horizontal line test and is therefore not a one-to-one function, its domain needs to be restricted.
In order to preserve all the outputs, the function needs to be restricted to the half-period which is ${\pi}$, as that is the minimum period where they are retained.
There are an infinite amount of ways this can be done; however, the best way is the domain ${x \in {[\frac{-\pi}{2}, \frac{\pi}{2}]}}$, as all the outputs are contained and is one-to-one, and it is the closest domain to the y-axis.
This is evident on the Graph, as the domain of ${y=\arcsin(x)}$ appears to be ${x \in {[{-1}, {1}]}}$ while the range is ${y \in {[\frac{-\pi}{2}, \frac{\pi}{2}]}}$, which are respectively the range and domain of ${y=\sin(x)}$.
The x and y intercept are both 0, meaning the point ${(0, 0)}$ belongs to the graph; this is reflected in ${y=\sin(x)}$, which has the point ${(0, 0)}$ as well.
