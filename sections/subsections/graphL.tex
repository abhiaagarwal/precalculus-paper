\subsection{Graph L}

Graph L is ${y=\arccos(x)}$.
Arccosine or Inverse Cosine is a function which takes a ratio as an input and outputs an angle.
Since ${y=\cos(x)}$ fails the horizontal line test and is therefore not a one-to-one function, its domain needs to be limited in order to make an inverse function.
In order to preserve all the outputs, the function needs to be restricted to the half-period which is ${\pi}$, as that is the minimum value where they all the unique outputs are retained.
Although there are an infinite (though limited) amount of ways this can be done, the best way is the domain ${x \in {[{0}, {\pi}]}}$, as all the outputs are contained and is one-to-one, and it is the closest domain to the y-axis that is most efficient.
This is evident, as the domain of ${y=\arcsin(x)}$ appears to be ${x \in {[{-1}, {1}]}}$ while the range is ${y \in {[{0}, {\pi}]}}$, which are respectively the range and domain of cos(x).
When Arccosine gets a negative ratio, it outputs an angle in $QII$, while receiving a positive ratio, it outputs an angle in $QI$.
This reflects cosine; when inputted with $QI$ angle, it gives a positive ratio, and with a $QII$ angle, a negative ratio.
