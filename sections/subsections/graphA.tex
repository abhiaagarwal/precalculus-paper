\subsection{Graph A}

Graph A is $y=\arccot(x)$.
Arccotangent or Inverse Cotangent gives an output whose angle resides in Quadrant I and II (and is therefore positive), and is the inverse function of $y=\cot(x)$.
The domain must be limited because the original function $y=\cot(x)$ fails the horizontal line test.
If one looks at the graph of $y=\cot(x)$, the graph has to have its domain restricted to a length of one period, $\pi$.
By doing this, one has obtained a function which retains all the outputs of the original function.
There are an infinite amount of ways to restrict the domain of $y=\cot(x)$ to the length of one period, however the best choice is $x \in {(0,\pi)}$.
This is because the function is continuous on this domain, while on other domains, the asymptotes of $x=0$, $x=\pi$ are included and will therefore leave “gaps” in the function, and it is the closest to the y-axis.
The range of the restricted function is still $y \in \R$.
An inverse function and a function share the property that their domain and ranges have been swapped, and this is evident with $y=\cot(x)$ and this graph.
The horizontal asymptotes for $y=\arccot(x)$ are $y=0$, $y=\pi$, and the vertical asymptotes $y=\cot(x)$ are $x=0$, $x=\pi$, and since $y=\arccot(x)$ is the inverse function of $y=\cot(x)$ from $x \in {(0,\pi)}$, the asymptotes will be reflected about $y=x$ as that is a property of an inverse function.
The y-intercept of the graph appears to be $\frac{\pi}{2}$, which is the y-intercept of $y=\arccot(x)$.
