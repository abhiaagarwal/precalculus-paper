\subsection{Graph H}

Graph H is $y=\tan(x)$.
Tangent is defined as the ratio of Sine and Cosine, and takes an angle and outputs a ratio.
The function appears to have a period of $3.14$, which is approximately $\pi$, a property of the function.
Since the function is defined as $\frac{\sin(x)}{\cos(x)}$, and ${\cos(n{\frac{\pi}{2}})=0}$ for ${\text{n is odd}, n\in \Z}$, tangent is undefined at all those values.
As such, it is asymptotic at $x={n{\frac{\pi}{2}}}$, which is evident on this graph.
${\sin(n\pi)=0}$ for ${n \in \Z}$, meaning that ${\tan(n{\pi})=0}$, so the x-intercepts of tangent are all integer multiples of ${\pi}$.
This is reflected on this graph.
During the first half of its period, ${\sin(x)}$ approaches $1$ from $0$, while ${\cos(x)}$ approaches $0$ from $1$, meaning that the function approaches $\infty$.
During the second half of its period, ${\sin(x)}$ approaches $0$ from $1$, while ${\cos(x)}$ approaches $-1$ from $0$, meaning the function approaches $0$.
